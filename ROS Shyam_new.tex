% Options for packages loaded elsewhere
\PassOptionsToPackage{unicode}{hyperref}
\PassOptionsToPackage{hyphens}{url}
%
\documentclass[
]{article}
\usepackage{amsmath,amssymb}
\usepackage{iftex}
\ifPDFTeX
  \usepackage[T1]{fontenc}
  \usepackage[utf8]{inputenc}
  \usepackage{textcomp} % provide euro and other symbols
\else % if luatex or xetex
  \usepackage{unicode-math} % this also loads fontspec
  \defaultfontfeatures{Scale=MatchLowercase}
  \defaultfontfeatures[\rmfamily]{Ligatures=TeX,Scale=1}
\fi
\usepackage{lmodern}
\ifPDFTeX\else
  % xetex/luatex font selection
\fi
% Use upquote if available, for straight quotes in verbatim environments
\IfFileExists{upquote.sty}{\usepackage{upquote}}{}
\IfFileExists{microtype.sty}{% use microtype if available
  \usepackage[]{microtype}
  \UseMicrotypeSet[protrusion]{basicmath} % disable protrusion for tt fonts
}{}
\makeatletter
\@ifundefined{KOMAClassName}{% if non-KOMA class
  \IfFileExists{parskip.sty}{%
    \usepackage{parskip}
  }{% else
    \setlength{\parindent}{0pt}
    \setlength{\parskip}{6pt plus 2pt minus 1pt}}
}{% if KOMA class
  \KOMAoptions{parskip=half}}
\makeatother
\usepackage{xcolor}
\usepackage{longtable,booktabs,array}
\usepackage{calc} % for calculating minipage widths
% Correct order of tables after \paragraph or \subparagraph
\usepackage{etoolbox}
\makeatletter
\patchcmd\longtable{\par}{\if@noskipsec\mbox{}\fi\par}{}{}
\makeatother
% Allow footnotes in longtable head/foot
\IfFileExists{footnotehyper.sty}{\usepackage{footnotehyper}}{\usepackage{footnote}}
\makesavenoteenv{longtable}
\usepackage{graphicx}
\makeatletter
\def\maxwidth{\ifdim\Gin@nat@width>\linewidth\linewidth\else\Gin@nat@width\fi}
\def\maxheight{\ifdim\Gin@nat@height>\textheight\textheight\else\Gin@nat@height\fi}
\makeatother
% Scale images if necessary, so that they will not overflow the page
% margins by default, and it is still possible to overwrite the defaults
% using explicit options in \includegraphics[width, height, ...]{}
\setkeys{Gin}{width=\maxwidth,height=\maxheight,keepaspectratio}
% Set default figure placement to htbp
\makeatletter
\def\fps@figure{htbp}
\makeatother
\setlength{\emergencystretch}{3em} % prevent overfull lines
\providecommand{\tightlist}{%
  \setlength{\itemsep}{0pt}\setlength{\parskip}{0pt}}
\setcounter{secnumdepth}{-\maxdimen} % remove section numbering
\ifLuaTeX
  \usepackage{selnolig}  % disable illegal ligatures
\fi
\IfFileExists{bookmark.sty}{\usepackage{bookmark}}{\usepackage{hyperref}}
\IfFileExists{xurl.sty}{\usepackage{xurl}}{} % add URL line breaks if available
\urlstyle{same}
\hypersetup{
  pdftitle={ROS 2--Based Dynamic Mapping, Path Planning, and Maze Solving for an Autonomous Mobile Robot},
  hidelinks,
  pdfcreator={LaTeX via pandoc}}

\title{\textbf{ROS 2--Based Dynamic Mapping, Path Planning, and Maze
Solving for an Autonomous Mobile Robot}}
\author{}
\date{}

\begin{document}
\maketitle

\hypertarget{abstract}{%
\section{\texorpdfstring{\textbf{Abstract}}{Abstract}}\label{abstract}}

Autonomous mobile robots are increasingly deployed in indoor and
semi-structured environments where reliable navigation is essential.
This paper presents the implementation and evaluation of a complete
autonomous navigation pipeline using ROS 2 in a simulated maze
environment. Dynamic mapping, probabilistic localization, and path
planning are achieved using SLAM Toolbox, AMCL, and the Nav2 framework.
In addition, a custom A*-based maze-solving algorithm is developed for
comparative evaluation. Simulation results demonstrate accurate mapping,
stable localization, and effective navigation with dynamic replanning
capabilities.

\emph{\textbf{Keywords:} ROS 2, SLAM, AMCL, Nav2, Autonomous Navigation,
Path Planning, Maze Solving}

\hypertarget{introduction}{%
\subsection{\texorpdfstring{\textbf{Introduction}}{Introduction}}\label{introduction}}

Autonomous mobile robots are increasingly deployed in indoor and
semi-structured environments such as warehouses, hospitals, and research
facilities, where safe and reliable navigation is critical. In such
environments, robots are required to operate without prior knowledge of
the surroundings, which necessitates the ability to perform mapping,
localization, and path planning in real time.

This coursework investigates the feasibility of implementing a complete
autonomous navigation pipeline using ROS 2 within a simulated maze
environment. The primary objective is to evaluate how effectively a
mobile robot can construct an environmental model, estimate its pose,
and navigate toward specified targets while responding to environmental
changes.

The study is conducted in simulation using Gazebo and RViz2, enabling
controlled experimentation while maintaining realism through
physics-based interactions. The robot platform selected is TurtleBot3,
which provides standardized sensors and software integration within ROS
2.

\hypertarget{problem-statement}{%
\subsubsection{\texorpdfstring{\textbf{Problem
Statement}}{Problem Statement}}\label{problem-statement}}

Autonomous mobile robots must be able to navigate from a start position
to a target location without human assistance. Simple follower-based or
straight-line motion approaches are not sufficient, as they do not
support goal-based navigation or global path planning. The objective of
this project is to develop an autonomous mobile robot that can compute
an optimal collision-free path using the A* algorithm and reach the goal
point autonomously. The system integrates mapping, localization, path
planning, and motion control under a Cyber-Physical Systems framework.

\hypertarget{state-of-the-art}{%
\subsubsection{\texorpdfstring{\textbf{State of the
Art}}{State of the Art}}\label{state-of-the-art}}

Various path planning techniques such as BFS, Dijkstra's algorithm, and
RRT have been proposed for autonomous navigation. However, these methods
either suffer from high computational complexity or do not guarantee
optimal paths. The A* algorithm is widely used in modern robotic systems
due to its efficiency and optimality achieved through heuristic-based
search. Robotic frameworks like ROS implement A* within navigation
stacks, combining it with localization and obstacle avoidance for
reliable autonomous navigation in indoor environments.

\hypertarget{methodology}{%
\subsubsection{\texorpdfstring{\textbf{Methodology}}{Methodology}}\label{methodology}}

The system begins with the generation of an occupancy grid map using
SLAM. The robot then localizes itself within the map and applies the A*
algorithm as a global planner to compute the optimal path to the goal. A
local control module follows the planned path by generating appropriate
velocity commands while avoiding obstacles. Continuous sensor feedback
enables closed-loop control, ensuring safe and autonomous navigation.

\hypertarget{potential-application-areas}{%
\subsubsection{\texorpdfstring{\textbf{Potential Application
Areas}}{Potential Application Areas}}\label{potential-application-areas}}

\textbf{4.1 Warehouse Automation}

Autonomous robots can efficiently transport goods within warehouses by
navigating predefined maps using A* path planning.

\textbf{4.2 Service Robots}

Service robots operating in hospitals, offices, and public spaces can
autonomously reach assigned locations for delivery and assistance tasks.

\textbf{4.3 Smart Manufacturing}

In smart factories, mobile robots can perform inspection and material
handling tasks by integrating autonomous navigation with cyber-physical
control systems.

\includegraphics[width=6.25087in,height=3.69843in]{media/image1.png}

\hypertarget{part-1-simulation-study-and-system-configuration}{%
\section{\texorpdfstring{\textbf{Part 1: Simulation Study and System
Configuration}}{Part 1: Simulation Study and System Configuration}}\label{part-1-simulation-study-and-system-configuration}}

\hypertarget{objective-of-simulation-investigation}{%
\subsection{\texorpdfstring{\textbf{Objective of Simulation
Investigation}}{Objective of Simulation Investigation}}\label{objective-of-simulation-investigation}}

The primary objective of this phase is to establish a representative
operational environment in which autonomous navigation algorithms can be
tested safely and repeatedly. Simulation is essential for validating
algorithmic performance prior to real-world deployment, reducing the
risk of hardware failure and enabling repeatable experimental
conditions.

\includegraphics[width=5.58218in,height=3.15669in]{media/image2.png}

The simulated environment consists of a structured maze containing
narrow corridors, junctions, and dead ends. This configuration provides
a challenging test scenario for mapping accuracy and path planning
efficiency.

\hypertarget{robot-model-and-sensor-configuration}{%
\subsection{\texorpdfstring{\textbf{Robot Model and Sensor
Configuration}}{Robot Model and Sensor Configuration}}\label{robot-model-and-sensor-configuration}}

The TurtleBot3 Burger platform is employed, which operates using a
differential drive mechanism. The robot is equipped with:

\begin{itemize}
\item
  Wheel encoders for odometry estimation
\item
  A 2D LiDAR sensor for obstacle detection and environment perception
\item
  Standardized coordinate frames for localization and navigation
\end{itemize}

Odometry data is derived from wheel rotations and is inherently subject
to cumulative drift. LiDAR provides accurate range measurements but may
suffer from reflective noise near sharp surfaces. The fusion of these
sensors forms the basis for both mapping and localization.

RViz2 is used to visualize the robot model, laser scans, coordinate
frames, and planned navigation paths, facilitating continuous system
verification.

\hypertarget{part-2-dynamic-mapping-and-environmental-modelling}{%
\section{\texorpdfstring{\textbf{Part 2: Dynamic Mapping and
Environmental
Modelling}}{Part 2: Dynamic Mapping and Environmental Modelling}}\label{part-2-dynamic-mapping-and-environmental-modelling}}

\hypertarget{slam-implementation-and-data-sources}{%
\subsection{\texorpdfstring{\textbf{SLAM Implementation and Data
Sources}}{SLAM Implementation and Data Sources}}\label{slam-implementation-and-data-sources}}

Simultaneous Localization and Mapping (SLAM) is implemented using SLAM
Toolbox, which employs scan matching and pose graph optimization to
estimate robot trajectory while generating an occupancy grid
representation of the environment.

The algorithm processes:

\begin{itemize}
\item
  LaserScan messages from the LiDAR sensor
\item
  Odometry data from wheel encoders
\item
  Transform relationships between coordinate frames
\end{itemize}

Through continuous motion, the robot incrementally constructs a
two-dimensional occupancy grid where each cell represents the
probability of being occupied.

\includegraphics[width=6.23004in,height=3.62551in]{media/image3.png}

This image represents the maze map generated dynamically while the robot
explores the environment. The map is saved as a .pgm file after SLAM
completes. Black areas indicate walls, white areas indicate free space,
and gray regions indicate unknown areas.

\hypertarget{analysis-of-generated-map}{%
\subsection{\texorpdfstring{\textbf{Analysis of Generated
Map}}{Analysis of Generated Map}}\label{analysis-of-generated-map}}

Figure 1 illustrates the final occupancy grid after full maze
exploration. Black regions correspond to wall structures and obstacles,
while white regions indicate navigable free space. Grey regions
represent unexplored or uncertain areas.

From the figure, it can be observed that corridor boundaries are well
defined, and structural symmetry of the maze is preserved. Minor
distortions near intersections indicate localized odometric drift, which
is expected in differential drive systems. However, loop closure events
reduce global distortion by realigning overlapping scan regions.

The resulting map is considered sufficiently accurate for localization
and navigation tasks, and is therefore saved and reused in subsequent
experimental phases.

\hypertarget{part-3-localization-performance-evaluation}{%
\section{\texorpdfstring{\textbf{Part 3: Localization Performance
Evaluation}}{Part 3: Localization Performance Evaluation}}\label{part-3-localization-performance-evaluation}}

\hypertarget{amcl-localization-technique}{%
\subsection{\texorpdfstring{\textbf{AMCL Localization
Technique}}{AMCL Localization Technique}}\label{amcl-localization-technique}}

Localization is performed using Adaptive Monte Carlo Localization
(AMCL), which applies particle filtering techniques to estimate robot
pose within a known map. Multiple hypotheses are maintained
simultaneously, and sensor measurements are used to weight particle
likelihood.

Particles with poor correspondence to sensor observations are discarded,
while high-likelihood particles are resampled, gradually converging
toward the true robot position.

\hypertarget{interpretation-of-localization-stability}{%
\subsection{\texorpdfstring{\textbf{Interpretation of Localization
Stability}}{Interpretation of Localization Stability}}\label{interpretation-of-localization-stability}}

Figure 3 displays the spatial distribution of particles around the robot
position. A dense clustering of particles indicates strong confidence in
pose estimation, whereas dispersed particles would suggest uncertainty
or localization instability.

From the figure, it is evident that particle concentration remains
localized even during robot movement, demonstrating stable localization.
This confirms that the map generated in Part 2 provides sufficient
structural features for reliable sensor matching.

\includegraphics[width=6.25087in,height=3.92763in]{media/image4.png}

\textbf{Figure 3: Particle Distribution During AMCL Localization}

Stable localization is critical for effective path planning, as
navigation algorithms depend on accurate robot pose estimation relative
to the environment.

\hypertarget{part-4-path-planning-and-navigation-assessment}{%
\section{\texorpdfstring{\textbf{Part 4: Path Planning and Navigation
Assessment}}{Part 4: Path Planning and Navigation Assessment}}\label{part-4-path-planning-and-navigation-assessment}}

\hypertarget{global-planning-using-nav2-framework}{%
\subsection{\texorpdfstring{\textbf{Global Planning Using Nav2
Framework}}{Global Planning Using Nav2 Framework}}\label{global-planning-using-nav2-framework}}

The Nav2 framework provides global and local planning capabilities using
layered costmaps and behavior trees. The global planner computes optimal
routes using graph search algorithms, while the local planner generates
real-time velocity commands to follow the path while avoiding obstacles.

\includegraphics[width=6.24045in,height=3.62551in]{media/image5.png}

\hypertarget{dynamic-objects-such-as-moving-obstacles-are-detected-in-real-time-using-lidar-and-are-shown-in-the-local-costmap.-these-obstacles-are-not-stored-in-the-static.-pgm-map-but-are-added-dynamically-to-the-costmap-layers-during-navigation.}{%
\subsection{Dynamic objects such as moving obstacles are detected in
real time using LiDAR and are shown in the local costmap. These
obstacles are not stored in the static. pgm map but are added
dynamically to the costmap layers during
navigation.}\label{dynamic-objects-such-as-moving-obstacles-are-detected-in-real-time-using-lidar-and-are-shown-in-the-local-costmap.-these-obstacles-are-not-stored-in-the-static.-pgm-map-but-are-added-dynamically-to-the-costmap-layers-during-navigation.}}

\hypertarget{custom-maze-solver-implementation}{%
\subsection{\texorpdfstring{\textbf{Custom Maze Solver
Implementation}}{Custom Maze Solver Implementation}}\label{custom-maze-solver-implementation}}

To assess algorithmic understanding beyond framework-level navigation, a
custom maze solver is implemented using the A* search algorithm. The
occupancy grid is converted into a discretized graph where each free
cell represents a node.

\includegraphics[width=6.25087in,height=3.91721in]{media/image6.png}

The A* algorithm evaluates both travel cost and heuristic distance to
the goal, producing efficient shortest paths while reducing
computational complexity.

The path is transmitted to a follower node that converts waypoints into
velocity commands using heading and distance calculations.

\hypertarget{comparative-path-evaluation}{%
\subsection{\texorpdfstring{\textbf{Comparative Path
Evaluation}}{Comparative Path Evaluation}}\label{comparative-path-evaluation}}

Figure 2 shows the global path between start and goal positions. The
trajectory remains centrally aligned within corridors and avoids
obstacle boundaries, minimizing collision risk and excessive turning.

\includegraphics[width=6.26806in,height=3.52569in]{media/image7.png}

Comparison between Nav2-generated paths and custom A*-generated paths
indicates similar routing patterns, validating the correctness of the
custom implementation. Minor deviations are attributed to smoothing
effects applied by local planners.

\includegraphics[width=5.96031in,height=3.97022in]{media/image8.png}

\hypertarget{part-5-motion-control-and-dynamic-response}{%
\section{\texorpdfstring{\textbf{Part 5: Motion Control and Dynamic
Response}}{Part 5: Motion Control and Dynamic Response}}\label{part-5-motion-control-and-dynamic-response}}

\hypertarget{differential-drive-motion-control}{%
\subsection{\texorpdfstring{\textbf{Differential Drive Motion
Control}}{Differential Drive Motion Control}}\label{differential-drive-motion-control}}

Motion control is implemented using proportional control on heading and
velocity. Angular velocity is adjusted according to angular error
relative to the next waypoint, while linear velocity is regulated based
on proximity to obstacles and path curvature with the dynamic maze.

\includegraphics[width=6.18971in,height=3.20833in]{media/image9.gif}

Speed limitations are applied to maintain stability in narrow corridors,
reducing collision probability. The maze. pgm is generated dynamically
using SLAM and represents the static environment. Incremental dynamic
objects are not stored in the .pgm file but are handled by the local
cost map using real-time sensor data, enabling dynamic replanning in the
Nav2 framework.

\hypertarget{dynamic-obstacle-handling-and-replanning}{%
\subsection{\texorpdfstring{\textbf{Dynamic Obstacle Handling and
Replanning}}{Dynamic Obstacle Handling and Replanning}}\label{dynamic-obstacle-handling-and-replanning}}

To evaluate dynamic response capability, obstacles are introduced after
initial planning. When obstacles intersect with planned routes, cost
maps update and global replanning is triggered automatically.

\includegraphics[width=6.25087in,height=2.75038in]{media/image10.png}

The robot successfully recalculates routes and avoids collisions,
demonstrating adaptability to environmental changes. This behavior is
essential for real-world deployment where static assumptions are
unrealistic.

\includegraphics[width=6.26806in,height=4.33333in]{media/image11.gif}

\hypertarget{when-a-dynamic-obstacle-blocks-the-planned-path-the-navigation-system-automatically-triggers-replanning.-the-global-planner-recomputes-a-new-path-using-a-while-the-local-planner-safely-avoids-the-obstacle.}{%
\subsection{When a dynamic obstacle blocks the planned path, the
navigation system automatically triggers replanning. The global planner
recomputes a new path using A* while the local planner safely avoids the
obstacle.}\label{when-a-dynamic-obstacle-blocks-the-planned-path-the-navigation-system-automatically-triggers-replanning.-the-global-planner-recomputes-a-new-path-using-a-while-the-local-planner-safely-avoids-the-obstacle.}}

\hypertarget{risk-factors-and-system-limitations}{%
\subsection{\texorpdfstring{\textbf{Risk Factors and System
Limitations}}{Risk Factors and System Limitations}}\label{risk-factors-and-system-limitations}}

\begin{longtable}[]{@{}
  >{\raggedright\arraybackslash}p{(\columnwidth - 8\tabcolsep) * \real{0.2000}}
  >{\raggedright\arraybackslash}p{(\columnwidth - 8\tabcolsep) * \real{0.2000}}
  >{\raggedright\arraybackslash}p{(\columnwidth - 8\tabcolsep) * \real{0.2000}}
  >{\raggedright\arraybackslash}p{(\columnwidth - 8\tabcolsep) * \real{0.2000}}
  >{\raggedright\arraybackslash}p{(\columnwidth - 8\tabcolsep) * \real{0.2001}}@{}}
\toprule\noalign{}
\begin{minipage}[b]{\linewidth}\raggedright
\textbf{Risk Factor}
\end{minipage} & \begin{minipage}[b]{\linewidth}\raggedright
\textbf{Source}
\end{minipage} & \begin{minipage}[b]{\linewidth}\raggedright
\textbf{Impact on System}
\end{minipage} & \begin{minipage}[b]{\linewidth}\raggedright
\textbf{Mitigation Strategy}
\end{minipage} & \begin{minipage}[b]{\linewidth}\raggedright
\textbf{Risk Factor}
\end{minipage} \\
\midrule\noalign{}
\endhead
\bottomrule\noalign{}
\endlastfoot
Odometry drift & Wheel slip & Pose estimation error & SLAM loop closure
& Odometry drift \\
Sensor noise & Reflective surfaces & Mapping inaccuracy & Scan filtering
& Sensor noise \\
Narrow passages & Maze geometry & Collision risk & Reduced velocity &
Narrow passages \\
Particle depletion & Localization failure & Navigation error &
Reinitialization & Particle depletion \\
\end{longtable}

\textbf{Table 1:} Identified System Risks and Mitigation Measures

\hypertarget{discussion}{%
\subsection{\texorpdfstring{\textbf{Discussion}}{Discussion}}\label{discussion}}

Results confirm that reliable autonomous navigation can be achieved
using ROS 2 and classical planning algorithms. Mapping quality directly
affects localization and navigation performance. While simulation offers
controlled evaluation, real-world deployment introduces additional
challenges such as hardware imperfections and environmental variability.

\hypertarget{conclusion}{%
\subsection{\texorpdfstring{\textbf{Conclusion}}{Conclusion}}\label{conclusion}}

This study demonstrates a complete ROS 2-based autonomous navigation
system in a simulated maze environment. Dynamic mapping, probabilistic
localization, and both framework-based and custom planning approaches
successfully solve navigation tasks. The system's ability to replan in
dynamic environments highlights its suitability for real-world robotic
applications.

\hypertarget{references}{%
\subsection{\texorpdfstring{\textbf{References}}{References}}\label{references}}

\textbf{{[}1{]}} D. Fox, W. Burgard, and S. Thrun, ``Monte Carlo
localization: Efficient position estimation for mobile robots,''
\emph{Proc. IEEE Int. Conf. Robotics and Automation (ICRA)}, Detroit,
MI, USA, 1999, pp. 343--349.

\textbf{{[}2{]}} S. Kohlbrecher, J. Meyer, T. Graber, K. Petersen, O.
von Stryk, and U. Klingauf, ``Hector SLAM: Real-time mapping and
localization for autonomous systems,'' \emph{Proc. IEEE/RSJ Int. Conf.
Intelligent Robots and Systems (IROS)}, Taipei, Taiwan, 2010, pp.
3988--3993.

\textbf{{[}3{]}} J. J. Leonard and H. F. Durrant-Whyte, ``Simultaneous
map building and localization for an autonomous mobile robot,''
\emph{Proc. IEEE/RSJ Int. Workshop on Intelligent Robots and Systems},
Osaka, Japan, 1991, pp. 1442--1447.

\textbf{{[}4{]}} S. Macenski, T. Moore, D. Lu, A. Merzlyakov, and M.
Ferguson, ``From the desks of ROS maintainers: A survey of modern
robotics frameworks,'' \emph{IEEE Robotics \& Automation Magazine}, vol.
27, no. 2, pp. 47--61, Jun. 2020.

\textbf{{[}5{]}} S. Macenski, M. Ferguson, W. Wise, J. Kunz, M. Walsh,
P. Bydalek, L. Gan, and M. Carson, ``The Navigation 2 framework: Design,
principles, and applications,'' \emph{Proc. IEEE/RSJ Int. Conf.
Intelligent Robots and Systems (IROS)}, Prague, Czech Republic, 2021,
pp. 3560--3566.

\textbf{{[}6{]}} R. Mur-Artal and J. D. Tardós, ``ORB-SLAM2: An
open-source SLAM system for monocular, stereo, and RGB-D cameras,''
\emph{IEEE Transactions on Robotics}, vol. 33, no. 5, pp. 1255--1262,
Oct. 2017.

\textbf{{[}7{]}} A. P. Narang, S. Verma, and P. Singh, ``Path planning
of mobile robot using A* algorithm,'' \emph{Int. Journal of Computer
Applications}, vol. 98, no. 12, pp. 36--40, Jul. 2014.

\textbf{{[}8{]}} L. E. Parker, ``Path planning and motion coordination
in multiple mobile robot teams,'' \emph{Encyclopedia of Complexity and
Systems Science}, Springer, 2009, pp. 5783--5800.

\textbf{{[}9{]}} M. Quigley et al., ``ROS: An open-source Robot
Operating System,'' \emph{Proc. IEEE Int. Conf. Robotics and Automation
(ICRA) Workshop}, Kobe, Japan, 2009.

\textbf{{[}10{]}} S. Thrun, W. Burgard, and D. Fox, \emph{Probabilistic
Robotics}, Cambridge, MA, USA: MIT Press, 2005.

\textbf{{[}11{]}} B. Yamauchi, ``A frontier-based approach for
autonomous exploration,'' \emph{Proc. IEEE Int. Symp. Computational
Intelligence in Robotics and Automation}, Monterey, CA, USA, 1997, pp.
146--151.

\textbf{{[}12{]}} Y. Zhang and S. Singh, ``LOAM: Lidar odometry and
mapping in real-time,'' \emph{Proc. Robotics: Science and Systems
(RSS)}, Berkeley, CA, USA, 2014.

\end{document}
